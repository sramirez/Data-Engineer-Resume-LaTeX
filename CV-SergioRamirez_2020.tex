%%%%%%%%%%%%%%%%%%%%%%%%%%%%%%%%%%%%%%%%%
% Twenty Seconds Resume/CV
% LaTeX Template
% Version 1.0 (14/7/16)
%
% Original author:
% Carmine Spagnuolo (cspagnuolo@unisa.it) with major modifications by 
% Vel (vel@LaTeXTemplates.com) and Harsh (harsh.gadgil@gmail.com)
%
% License:
% The MIT License (see included LICENSE file)
%
%%%%%%%%%%%%%%%%%%%%%%%%%%%%%%%%%%%%%%%%%


%% BUILD with XELATEX, install texlive-full, and replace faEnvelopeAlt by faEnvelope

%----------------------------------------------------------------------------------------
%	PACKAGES AND OTHER DOCUMENT CONFIGURATIONS
	%----------------------------------------------------------------------------------------

\documentclass[letterpaper]{twentysecondcv} % a4paper for A4

% Command for printing skill overview bubbles
\newcommand\skills{ 
~
	\smartdiagram[bubble diagram]{
        \textbf{Data}\\\textbf{Science},
        \textbf{Data}\\\textbf{Wrangling},
        \textbf{Autonomous}\\\textbf{R+D},
        \textbf{Commun.}\\\textbf{English (C1)},
        \textbf{Machine}\\\textbf{Learning},
        \textbf{Data}\\\textbf{Visualization},
        \textbf{Statistical}\\\textbf{Analysis}
    }
}

% Programming skill bars
\programming{{C $\textbullet$ Docker $\textbullet$ Cloud $\textbullet$ Linux scripting / 2}, {Java $\textbullet$ Git $\textbullet$ Flink $\textbullet$ R / 4}, {Scala $\textbullet$ Spark $\textbullet$ Python / 6}}

% Projects text
\software{
\textbf{Spark-IT-FS*} - A distributed feature selection framework for Apache Spark | 111 \faStar\\
\textbf{fast-mRMR} - mRMR feature selector for Spark | 57 \faStar\\
\textbf{Spark-MDLP*} - A Spark implementation of the MDLP discretizer (presented in Spark Summit '17) | 39 \faStar\\
\textbf{FeuerFreiKiller} - A deep learning project to automatically detect wildfire in nature using spectogram images.
%\textbf{MOAReduction} - A library for streaming data reduction in MOA | 2 \faStar\\
%\textbf{Spark-IS-streaming} A nearest neighbor classifier for high-speed big data streams with instance selection | 1 \faStar\\\\
\\
\textbf{*}\url{https://spark-packages.org/}.
}
%----------------------------------------------------------------------------------------
%	 PERSONAL INFORMATION
%----------------------------------------------------------------------------------------
% If you don't need one or more of the below, just remove the content leaving the command, e.g. \cvnumberphone{}

\cvname{Sergio Ramírez Gallego} % Your name
\cvjobtitle{Machine Learning Engineer} % Job
%Born in 1988 in Jaén, Spain.
% title/career

\aboutme{Born in 1988}
\cvlinkedin{/in/sramirezg}
\cvgithub{sramirez}
\cvnumberphone{(34) 699 72 40 43} % Phone number
\cvsite{} % Personal website
\cvmail{srg00017@gmail.com} % Email address

%----------------------------------------------------------------------------------------

\begin{document}

\makeprofile % Print the sidebar



%----------------------------------------------------------------------------------------
%	 EXPERIENCE
%----------------------------------------------------------------------------------------

\section{Experience}
\begin{twenty} % Environment for a list with descriptions

\twentyitem
    	{Jan 2020 - }
		{Now}
        {Machine Learning Engineer}
        {\href{https://pragsis.com/}{Pragsis Bidoop, Part of Accenture Applied Intelligence}}
        {}
        {\begin{itemize}
			\item Projects: Crowd behaviour analysis from drone view.
			\item Creation of several computer vision models with Deep Learning. Crowd segmentation with EfficientNet+UNet (CNN). Object detection with YOLOv3 and IOU tracking. Human pose estimation with AlphaPose. Dataset creation and annotation.
			\item \textbf{Stack:} OpenCV2, Keras, PyTorch, Tensorflow, Albumentations, Pillow, UNet, CVAT annotation tool.
        \end{itemize}}
        \\
\twentyitem
    	{Nov 2019 - }
		{Now}
        {Lead Machine Learning Engineer}
        {\href{https://omdena.com}{Omdena}}
        %{\href{https://github.com/omdena/ImpactHubIstanbul/tree/master/notebooks/satellital}{Omdena Earthquake challenge}}
        {}
        {\begin{itemize}
			\item Project: Improving the aftermath management of an earthquake with AI.
			\item Deep Learning model (CNN) to automatically detect buildings footprints from satellite imagery. Data extraction from Mapbox API. Post-processing: creation of a risk heatmap with distance to dangerous buildings. Definition of the final solution.
			\item {\href{https://medium.com/omdena/estimating-street-safeness-after-an-earthquake-with-deep-learning-f2ae50b9e25e}{\textbf{[LINK]}}} Medium article.
			\item \textbf{Stack:} geopandas, geojson, GDAL, fast.ai, PyTorch, Solaris.
        \end{itemize}}
        \\
\twentyitem
    	{Mar 2019 - }
		{Jan 2020}
        {Machine Learning Engineer}
        {\href{https://www.docomodigital.com/}{NTT Docomo Digital Spain Inc.}}
        {}
        {\begin{itemize}
        \item Projects: Default prediction in mobile carrier billing. 
        \item Tasks: Scala production-ready code for all stages in the ML pipeline. Data analysis and modeling with XGBoost. Development of a proper model validation schema for uncertain and delayed labeling in imbalanced classification. Production deployment, unit testing, scaling and performance tuning. Product definition in cooperation with the business team. Interviewed junior roles.
        \item \textbf{Stack:} Spark, Scala, Python, Cloudera, HDFS, MongoDB, Jenkins, LDA, XGBoost, LightGBM, Kibana.
        \end{itemize}}
        \\
\twentyitem
    	{Jun 2018 -}
		{Mar 2019}
        {Data Scientist \& Big Data Developer}
        {\href{http://www.stratio.com/}{Stratio Big Data Inc.}}
        {}
        {\begin{itemize}
        \item Projects: Default prediction in mobile carrier billing with Spark and XGBoost. Customer churn prediction of millions of customers with PySpark (Databricks). Ads optimization in social media with reinforcemet learning (Thompson sampling, OpenAI gym). 
        \item \textbf{Stack:} Spark, Scala, Python, MongoDB, Jenkins, XGBoost, Kibana. 
        \end{itemize}}
        \\     %\\
        %{\begin{itemize}
        %\item FPU Spanish Research and Teaching Fellowship @ Department of Computer Science and Artificial Intelligence.
        %\end{itemize}}

% \twentyitem
%    	{Dec 2016 -}
%		{May 2017}
%        {Data Scientist}
%        {\href{http://www.cetaqua.com/en/cetaqua}{University of Granada \& CETAQUA foundation}}
%        {}
%        {\begin{itemize}
%        	\item Time-series forecasting applied to water-consumption in Spanish cities. \textbf{Tools}: %R, ggplot2, forecast, tsExpKit, caret, tidyr, knitr.
%        \end{itemize}}
        
%\\
 %\twentyitem
 %   	{Oct 2013 -}
%		{Mar 2015}
%        {Data Scientist}
%        {\href{http://www.granadalapalma.com/}{University of Granada \& LaPalma Cooperative}}
%        {}
%        {\begin{itemize}
%        	\item Time-series forecasting for greenhouse farming. \textbf{Tools}: R, ggplot2, forecast, tsExpKit, caret, tidyr, knitr.
%        \end{itemize}}
        
%\\
        
	  %\twentyitem
    	%{Apr 2013 -}
		%{Dec 2013}
        %{Data Scientist}
        %{\href{https://www.ugr.es/}{ADIF, University of Granada and others.}}
        %{}
        %{\begin{itemize}
       % 	\item Time-series forecasting of railway deficiencies. %Multi-objective optimization for railway infrastructure maintenance. In cooperation with VIAS Inc., ADIF, and others.
        	%En GEOMAF se hizo una predicción de los fallos en vías de tren usando series temporales, con R y Java y usando PMML como interfaz para portar modelos de un lenguaje a otro. En OPTIRAIL hicimos generación de planes de mantenimiento de infraestructuras ferroviarias con algoritmos multiobjetivo, en R y en C++. YO ME CENTRÉ en un modelo del deterioro de las vías en función del tiempo y del mantenimiento que se le aplica, con un exponential fitting.
        	%\textbf{Tools}: R, Java, PMML, C++.
        %\end{itemize}}
	  \twentyitem
    	{Dec 2011 -}
		{Mar 2018}
        {Data Scientist and Research Consultant}
        {\href{https://www.indracompany.com/en/}{University of Granada}}
        {
        \begin{itemize}
        	\item Projects:
        	\begin{itemize}
        		\item \textbf{\textit{Indra Systems}}: Workload prediction in cloud systems with Google-based data. Development of an extension for big data development in Eclipse IDE.
        		\item \textbf{\textit{ADIF and others}}: Time-series forecasting of railway deficiencies. Multi-objective optimization for train infrastructures.
        		\item \textbf{\textit{LaPalma cooperative}}: Time-series forecasting of greenhouse climate conditions. Automatic code generation of analytic plots with ggplot2.
        		\item \textbf{\textit{CETAQUA foundation}}: Time-series forecasting applied to water consumption in Spanish cities. Data analysis and modeling.
        	\end{itemize}
        	\item \textbf{Stack:} R, ggplot2, forecast, tsExpKit, caret, tidyr, knitr.
        
        \end{itemize}
        }
        {}  
        %\begin{itemize}
        %	\item Development of a Eclipse plugin to deploy Java apps and manage non-relational DBs in APlaCA, a cloud computing platform. Time-series forecasting of platform load. \textbf{Tools}: Java, Eclipse, R, HTML.
        %\end{itemize}
        %%\twentyitem
    	%{Dec 2011 -}
		%{Mar 2012}
        %{Data Scientist}
        %{\href{http://en.orolivesur.com/}{University of Jaen \& OroLiveSur Inc.}}
        %{}
        %{}
        %{\begin{itemize}
        %	\item Extraction and analysis of user-patterns in an e-commerce site. Rule learning applied to user-patterns. \textbf{Tools}: Java, Weka, KEEL.
        %\end{itemize}}
        %\\
	%\twentyitem{<dates>}{<title>}{<location>}{<description>}
\end{twenty}

\newpage
\makeprofile % Print the sidebar
\\

%----------------------------------------------------------------------------------------
%	 EDUCATION
%----------------------------------------------------------------------------------------
\section{Education}

\begin{twenty} % Environment for a list with descriptions+
	\twentyitem
    	{2018 - Now}
        {}
        {Self-learning}
        {}
        {\textbf{Courses}: \begin{itemize}
        	\item Coursera Deep Learning Specialization (by deeplearning.ai): CNN, Sequence Models, Structuring ML projects, NNs, Improving Deep NNs.
			\item Natural Language Processing (by National Research University Higher School of Economics). Final project: Telegram chatbot (chitchat + answering from StackOverflow). Deployed in AWS EC2.  
			\item How to Win a Data Science Competition: Learn from Top Kagglers (by National Research University Higher School of Economics).
			\item Open Data Science free-course (by ods.ai).
        \end{itemize}}
        {\textbf{Challenges and competitions}:
        	\begin{itemize}
        		\item Omdena challenge: Improving the Aftermath Management of an Earthquake with AI.
        		\item Kaggle competition: Understanding Clouds from Satellite Images, among others.
        	\end{itemize}
        }
       
	\twentyitem
    	{2013 - 2018}
        {}
        {Ph. D., Artificial Intelligence \textnormal{CUM LAUDE}}
        {\href{http://www.ugr.es/}{University of Granada, Spain}}
        {{\includegraphics[scale=0.04]{img/trophy.png}} Best Ph. D. Thesis Project award by Spanish Association of
Artificial Intelligence (AEPIA)}
        {}
	
	\twentyitem
    	{2012 - 2013}
        {}
        {MSc., Soft Computing and Intelligent Systems \textnormal{(GPA: 8.1/10.0)}}
        {\href{http://www.ugr.es/}{University of Granada, Spain}}
        {}
        {}
	\twentyitem
    	{2009 - 2012}
		{}
        {BEng., Computer Engineering \textnormal{(GPA: 8.8/10.0)}}
        {\href{http://www.ujaen.es/}{University of Jaen, Spain}}
        {}
        {}
   \twentyitem
    	{2006 - 2009}
		{}
        {BEng., Computer Technical Engineering \textnormal{(GPA: 7.9/10.0)}}
        {\href{http://www.ujaen.es/}{University of Jaen, Spain}}
        {{\includegraphics[scale=0.04]{img/trophy.png}} Bachelor extraordinary award}
        {}
	%\twentyitem{<dates>}{<title>}{<organization>}{<location>}{<description>}
\end{twenty}

%----------------------------------------------------------------------------------------
%	 RESEARCH
%----------------------------------------------------------------------------------------


\section{Research}
\begin{twenty}
	\twentyitem
    	{2012 - 2018}
		{}
        {Ph. D. Candidate, Graduate Research and Teaching Assistant}
        {\href{http://www.ugr.es/}{University of Granada}}
        {}
        {
       	\textbf{Thesis}: Distributed Data Reduction Algorithms for Big Data
        {\begin{itemize}
        \item Scalable data reduction (discretization, feature and instance selection) algorithms for Big Data with Apache Spark. Fast feature selection on 30M cols (kdd2010), and discretization on 60M rows.
        %\item Proposed an scalable method that reduces noisiness in databases fed by high-speed big data steams. Our algorithm (developed on Spark Streaming) maintains a polished case-base with the most relevant instances seen so far.
        \item \textbf{Tools}: Spark, Scala, Java, MOA, Flink, \LaTeX, Git, Maven.
        \item \textbf{Internships}: Wroclaw University of Science and Technology, Poland \& M\"alardalen University, V\"aster\aa s, Sweden. Duration: 3 months each.        
        \item \textbf{Publications}: \url{https://scholar.google.com/citations?user=cEkfaW4AAAAJ}
        \item More than 1K cites in several papers published in top-tier international scientific journals.
        \item \textbf{Books}: \href{https://link.springer.com/book/10.1007/978-3-030-39105-8}{Big Data Preprocessing}. 
		\end{itemize}}
        }
\end{twenty}
\end{document} 
