%%%%%%%%%%%%%%%%%%%%%%%%%%%%%%%%%%%%%%%%%
% Twenty Seconds Resume/CV
% LaTeX Template
% Version 1.0 (14/7/16)
%
% Original author:
% Carmine Spagnuolo (cspagnuolo@unisa.it) with major modifications by 
% Vel (vel@LaTeXTemplates.com) and Harsh (harsh.gadgil@gmail.com)
%
% License:
% The MIT License (see included LICENSE file)
%
%%%%%%%%%%%%%%%%%%%%%%%%%%%%%%%%%%%%%%%%%

%----------------------------------------------------------------------------------------
%	PACKAGES AND OTHER DOCUMENT CONFIGURATIONS
	%----------------------------------------------------------------------------------------

\documentclass[letterpaper]{twentysecondcv} % a4paper for A4

% Command for printing skill overview bubbles
\newcommand\skills{ 
~
	\smartdiagram[bubble diagram]{
        \textbf{Data}\\\textbf{Science},
        \textbf{Data}\\\textbf{Wrangling},
        \textbf{Autonomous}\\\textbf{R+D},
        \textbf{Commun.}\\\textbf{English (C1)},
        \textbf{Machine}\\\textbf{Learning},
        \textbf{Data}\\\textbf{Visualization},
        \textbf{Statistical}\\\textbf{Analysis}
    }
}

% Programming skill bars
\programming{{C $\textbullet$ C++ $\textbullet$ Python $\textbullet$ Maven / 3}, {Java $\textbullet$ Git $\textbullet$ Flink $\textbullet$ \large \LaTeX / 4}, {Scala $\textbullet$ Spark $\textbullet$ R / 5}}

% Projects text
\software{
\textbf{Spark-IT-FS} - A distributed feature selection framework for Apache Spark | 82 \faStar\\
\textbf{fast-mRMR} - An optimized version of the mRMR feature selector for Spark and Nvidia-GPU | 34 \faStar\\
\textbf{Spark-MDLP} - A Spark implementation of the MDLP discretizer (presented in Spark Summit '17) | 29 \faStar\\
\textbf{MOAReduction} - A library for streaming data reduction in MOA | 2 \faStar\\
%\textbf{Spark-IS-streaming} A nearest neighbor classifier for high-speed big data streams with instance selection | 1 \faStar\\\\
\\
\emph{Check my GitHub profile, or my Spark packages:} \url{https://spark-packages.org/user/sramirez}.
}
%----------------------------------------------------------------------------------------
%	 PERSONAL INFORMATION
%----------------------------------------------------------------------------------------
% If you don't need one or more of the below, just remove the content leaving the command, e.g. \cvnumberphone{}

\cvname{Sergio Ramírez Gallego} % Your name
\cvjobtitle{ Data Scientist } % Job
%Born in 1988 in Jaén, Spain.
% title/career

\aboutme{Born in 1988 in Jaén, Spain} %Luis de Narváez, 7 3L\\18014 Granada, Spain.}
\cvlinkedin{/in/sramirezg}
\cvgithub{sramirez}
\cvnumberphone{(34) 699 72 40 43} % Phone number
\cvsite{} % Personal website
\cvmail{sramirez@decsai.ugr.es} % Email address

%----------------------------------------------------------------------------------------

\begin{document}

\makeprofile % Print the sidebar

%----------------------------------------------------------------------------------------
%	 EDUCATION
%----------------------------------------------------------------------------------------
\section{Education}

\begin{twenty} % Environment for a list with descriptions
	\twentyitem
    	{2013 - 2018}
        {}
        {Ph. D., Computer Science \textnormal{CUM LAUDE}}
        {\href{http://www.ugr.es/}{University of Granada, Spain}}
        {{\includegraphics[scale=0.04]{img/trophy.png}} Best Ph. D. Thesis Project award by Spanish Association of
Artificial Intelligence (AEPIA)}
        {}
	
	\twentyitem
    	{2012 - 2013}
        {}
        {MSc., Soft Computing and Intelligent Systems \textnormal{(GPA: 8.1/10.0)}}
        {\href{http://www.ugr.es/}{University of Granada, Spain}}
        {}
        {}
	\twentyitem
    	{2009 - 2012}
		{}
        {BEng., Computer Engineering \textnormal{(GPA: 8.8/10.0)}}
        {\href{http://www.ujaen.es/}{University of Jaen, Spain}}
        {}
        {}
   \twentyitem
    	{2006 - 2009}
		{}
        {BEng., Computer Technical Engineering \textnormal{(GPA: 7.9/10.0)}}
        {\href{http://www.ujaen.es/}{University of Jaen, Spain}}
        {{\includegraphics[scale=0.04]{img/trophy.png}} Bachelor extraordinary award}
        {}
	%\twentyitem{<dates>}{<title>}{<organization>}{<location>}{<description>}
\end{twenty}

%----------------------------------------------------------------------------------------
%	 RESEARCH
%----------------------------------------------------------------------------------------


\section{Research}
\begin{twenty}
	\twentyitem
    	{2012 - 2018}
		{}
        {Ph. D. Candidate, Graduate Research Assistant}
        {\href{http://www.ugr.es/}{University of Granada}}
        {}
        {
       	\textbf{Thesis}: Distributed Data Reduction Algorithms for Big Data
        {\begin{itemize}
        \item Proposed several scalable data reduction (discretization, feature and instance selection) techniques for Apache Spark. Aimed at decreasing the complexity of large-scale static and dynamic DBs.
        %\item Proposed an scalable method that reduces noisiness in databases fed by high-speed big data steams. Our algorithm (developed on Spark Streaming) maintains a polished case-base with the most relevant instances seen so far.
        \item \textbf{Tools}: Spark, Scala, Java, MOA, Flink, \LaTeX, Git, Maven, Eclipse.
        %\item \textbf{Internships}: PWr, Wroclaw, Poland (3m), MDH, Sweden (3m).         
        \item Summary of publications: \url{https://scholar.google.com/citations?user=cEkfaW4AAAAJ}
		\end{itemize}}
        }
\end{twenty}

%----------------------------------------------------------------------------------------
%	 EXPERIENCE
%----------------------------------------------------------------------------------------

\section{Experience}



\begin{twenty} % Environment for a list with descriptions

\twentyitem
    	{Sep 2014 -}
		{Feb 2018}
        {Researcher \& Graduate Teaching Assistant}
        {\href{http://www.ugr.es/}{University of Granada}}
        {}
        {\begin{itemize}
        \item FPU Spanish Research and Teaching Fellowship @ Department of
Computer Science and Artificial Intelligence.
        \end{itemize}}
        \\
 \twentyitem
    	{Dec 2016 -}
		{May 2017}
        {Data Scientist}
        {\href{http://www.cetaqua.com/en/cetaqua}{University of Granada \& CETAQUA foundation}}
        {}
        {\begin{itemize}
        	\item Water-consumption forecasting in Spanish eastern cities. Data analysis, visualization and time-series forecasting. \textbf{Tools}: R, ggplot2, forecast, tsExpKit, knitr.
        \end{itemize}}
        
\\
 \twentyitem
    	{Oct 2013 -}
		{Mar 2015}
        {Data Scientist}
        {\href{http://www.granadalapalma.com/}{University of Granada \& LaPalma Cooperative}}
        {}
        {\begin{itemize}
        	\item Crop forecasting and detection of plagues in greenhouses using weather conditions. Data analysis, visualization and time-series forecasting. \textbf{Tools}: R, ggplot2, forecast, tsExpKit, knitr.
        \end{itemize}}
        
\\
        
	  \twentyitem
    	{Apr 2013 -}
		{Dec 2013}
        {Data Scientist}
        {\href{https://www.ugr.es/}{University of Granada}}
        {}
        {\begin{itemize}
        	\item Time-series forecasting of railway deficiencies. Multi-objective optimization for railway infrastructure maintenance. In cooperation with VIAS Inc., ADIF, and others.
        	%En GEOMAF se hizo una predicción de los fallos en vías de tren usando series temporales, con R y Java y usando PMML como interfaz para portar modelos de un lenguaje a otro En OPTIRAIL hicimos generación de planes de mantenimiento de infraestructuras ferroviarias con algoritmos multiobjetivo, en R y en C++También hicimos un modelo del deterioro de las vías en función del tiempo y del mantenimiento que se le aplica, con un exponential fitting 
        	\textbf{Tools}: R, Java, PMML, C++.
        \end{itemize}}
        
\\
	  \twentyitem
    	{Mar 2012 -}
		{Apr 2013}
        {Full Stack Developer \& Data Scientist}
        {\href{https://www.indracompany.com/en/}{Univ. of Granada \& Indra Systems}}
        {}
        {\begin{itemize}
        	\item Development of a Eclipse plugin to code/deploy Java apps, and manage non-relational DBs in APlaCA, a cloud platform. Time-series forecasting of platform load. \textbf{Tools}: Java, Eclipse, R, HTML.
        \end{itemize}}
        \\
        \twentyitem
    	{Dec 2011 -}
		{Mar 2012}
        {Data Scientist}
        {\href{http://en.orolivesur.com/}{University of Jaen \& OroLiveSur Inc.}}
        {}
        {\begin{itemize}
        	\item Extraction and analysis of user-patterns in an e-commerce site. Rule learning from user-patterns. \textbf{Tools}: Java, Weka, KEEL.
        \end{itemize}}
        \\
    \twentyitem
   		{Sep 2013 -}
		{Feb 2018}
        {Linux Server Administrator}
        {\href{http://sci2s.ugr.es}{SCI2S Group, Dept. of Computer Science and AI, UGR}}
        {}
        {
        {\begin{itemize}
        \item \textbf{Tools}: CentOS, LDAP, SGE, Spark, Hadoop, Ganglia, KickStart, PXE.
        %\item TA for \textit{Data Structures}, \textit{Design and Development of Information Systems} courses (practices).
    \end{itemize}}
        }
	%\twentyitem{<dates>}{<title>}{<location>}{<description>}
\end{twenty}

\end{document} 
